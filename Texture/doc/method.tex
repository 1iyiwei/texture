\section{Method}
\label{sec:method}

Written words are more concrete than ideas in your head.
Math equations and pseudo-codes are even more concrete than words, and can help, even though not essential, for exposition.
Programs that produce the actual results are the most concrete, but the amount and level of information is not suitable for a research paper.
I like to write down math and pseudo-code first as a way to design the algorithms before actual implementation.

\subsection{Math}
\label{sec:math}

\begin{align}
\energy = \mass \lightspeed^2
\label{eqn:emc2}
\end{align}

Always use defined instead of naked math symbols for clarify and consistency.
For example, in \Cref{eqn:emc2} I define all symbols inside \filename{symbols.tex} so later if I need to change $\energy$ from $E$ to $e$ I just need to change one line in \filename{symbols.tex} instead of chasing $E$ everywhere.
This can save you a lot of time and sanity if you have many math symbols.

\subsection{Code}
\label{sec:code}

\begin{algorithm}
  \begin{algorithmic}[1]
    \REQUIRE sample domain $\domainsym$ with distance $\dist$ and conflict $\conflictdist$ measure
    \ENSURE output sample set $\sampleset$
    \STATE $\sampleset \assign \emptyset$
    \STATE $\failure \assign 0$ \pcomment{failure count}
    \WHILE{$\domainsym$ is not fully covered and $\failure$ not too high}
    \STATE $\sample \assign \funct{RandomSample}(\domainsym)$
    \IF{$\dist(\sample, \sampleprime) < \conflictdist(\sample, \sampleprime) \; \forall \sampleprime \in \sampleset$}
    \STATE $\sampleset \assign \sampleset \union \sample$ \pcomment{accept trial sample}
    \STATE $\failure \assign 0$
    \ELSE
    \STATE $\failure \assign \failure + 1$
    \ENDIF
    \ENDWHILE
    \RETURN{} $\sampleset$
  \end{algorithmic}
  \Caption{Dart throwing.}
  {%
This is an example of pseudo-code.
  }
  \label{alg:dart_throw}
\end{algorithm}




Pseudo-code can help summarize and explain the algorithms, but only if presented with sufficient clarity and simplicity. 
\Cref{alg:dart_throw} is an example.
All algorithms should be understandable from the main texts without looking at the pseudo-codes, which are usually more suitable for summarization than explanation.

