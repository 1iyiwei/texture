\section{Implementation}
\label{sec:implementation}

The current codebase is implemented in c++ consisting of the following main groups of classes:

\begin{description}

\item[Image and Texture]

These are arrays holding pixels and texels for images and textures.
A texture holds source position for each texel in addition to image pixel.

\item[Synthesizer]

A synthesizer decides the color for a given output texel.
\classname{RandomSynthesizer} simply picks a random candidate from the input and can serve as initialization.
\classname{CoherenceSynthesizer} implements the main method described in \Cref{sec:algorithm}.

\item[Sequencer]

A sequencer decides the traversal order of synthesizing output texels.
\classname{ScanlineSequencer} traverses in a raster order.
\classname{RandomShuttleSequencer} and \classname{RandomVisitSequencer} traverse in a random order with and without guarantee of once and only once per output texel.
\classname{RandomDiffusionSequencer} starts from the constraint points and gradually moves outwards.
 
\item[Neighborhood and Template]
A template decides the neighborhood size and shape, which is then fed into a neighborhood class for various related computations such as comparing two neighborhoods.

\classname{SquareTemplate} is the basic square shape.
\classname{CausalTemplate} converts any given template for causality, mainly in combination with \classname{ScanlineSequencer}.

\classname{PlainNeighborhood} regards out of bound neighbors as null, and \classname{ToroidalNeighborhood} wraps up toroidally which is useful for producing tileable outputs.

\item[Pyramid]

These are the multi-resolution counterparts for the basic classes above, e.g. \classname{TexturePyramid}/\classname{PyramidNeighborhood} as the multi-resolution extension of \classname{Texture}/\classname{Neighborhood}.

\item[Utility and Synth]

The \classname{Utility} class contains the main code for organizing synthesis, while \classname{Synch} is its dumb main.

\end{description}

