\section{Method}
\label{sec:method}
\label{sec:algorithm}

\paragraph{Assumption}

The key assumption about textures is locality under contextual control \cite{Wei:2009:STTS}.
In particular, each texture sample (e.g. pixel of an image texture or vertex of a mesh texture) relates to only those that are in its spatial vicinity (locality) \cite{Efros:1999:TSN}, but its property is conditioned upon global control such as environment variables (context) \cite{Lu:2007:CT}.
Taking these two together, a texture consists of repetitive patterns that are local but potentially globally varying.

\paragraph{Idea}

Our method is adapted from 
k-coherence \cite{Tong:2002:SBT} and randomized patch-match \cite{Barnes:2009:PRC} for quality\nothing{efficacy}, efficiency, flexibility, and simplicity.
The basic idea stems from the coherence observation: regardless of the exact synthesis algorithms used, as long as the output has sufficient quality, it is usually composed of coherent patches from the input exemplar.
In other words, if we record the source position $\position(\inputex, \sample)$ within the input exemplar $\inputex$ from which each output pixel $\sample \in \outputex$ comes from, the positions usually form coherent patches as visualized in \cite{Wei:2002:TSF}.
This implies that to determine the color value $\range(\sample)$ of $\sample$, we can look at each one of its already synthesized neighbor $\sample' \in \outputex$ with input position $\position(\inputex, \sample')$, and use their relative input and output position offsets to compute a good potential candidate:
\begin{align}
\position(\inputex, \sample' \rightarrow \sample) = \position(\outputex, \sample) - \position(\outputex, \sample') + \position(\inputex, \sample')
\label{eqn:coherence_offset}
\end{align}
, where $\position(\inputex, \sample' \rightarrow \sample)$ indicates a potentially good source position of $\sample$ at the input $\inputex$ as voted from an already computed output pixel $\sample'$.
We can collect a set of such $\sample'$ in the (coherence) neighborhood $\corneighborhood$ of $\sample$, and see which one has the best matched (texture) neighborhood $\texneighborhood$:
\begin{align}
\range(\outputex, \sample) \leftarrow \range\left(\argmin_{\sample'} \left\{\left|\texneighborhood(\outputex, \sample) - \texneighborhood(\outputex, \sample')\right|^2 \forall \sample' \in \corneighborhood(\outputex, \sample) \right\} \right)
\label{eqn:coherence_synthesis}
\end{align}
The coherence and texture neighborhoods $\corneighborhood$ and $\texneighborhood$ are usually small regions with specific shapes and sizes, e.g. circles or squares.

\paragraph{Variations}

Both k-coherence \cite{Tong:2002:SBT} and patch-match \cite{Barnes:2009:PRC} are generalizations of the main idea expressed in \Cref{eqn:coherence_synthesis}.
In particular, \cite{Tong:2002:SBT} pre-computes a set of $\knumber$ candidates with similar texture neighborhoods to each input pixel $\sample \in \inputex$, so that each $\sample$ can provide $\knumber$ instead of just $1$ votes in \Cref{eqn:coherence_offset}.
Empirical evidence indicates that this can indeed improve quality (with sufficiently small $\knumber$ in the range of 4 to 8), but this pre-computation can be too expensive for large input exemplars.
Patch-match \cite{Barnes:2009:PRC} also bypasses this pre-computation, but instead of adding $\knumber$ random candidates to \Cref{eqn:coherence_synthesis}, which works surprisingly well.

Optimization \cite{Kwatra:2005:TOE,Han:2006:TDO} can be imposed as an orthogonal component, but greedy assignment usually suffices.

\paragraph{Constraint}

Constraints can be easily added by simply presetting the colors of certain output pixels in \Cref{eqn:coherence_synthesis}, and let the synthesis process assign neighboring output to have consistent texture patterns.
Good output quality can never be fully guaranteed as it depends on several factors, including whether the constraints leave enough degrees of freedom for optimization.

\paragraph{Multi-resolution}

The basic ideas above can be directly extended for multi-resolution synthesis \cite{Wei:2000:FTS} via the following basic steps:
\begin{enumerate}
\item
Compute a Gaussian pyramid from the input with number of levels specified by the user.

\item
Synthesize the output from lower to higher levels, with each level generated from the corresponding input level.
Each level computation is very similar to the single-resolution algorithm above, except that we can use multi-resolution neighborhoods (for both similarity measure and coherence).

\item
Write out the highest output level.
\end{enumerate}

Hard constraints are currently handled directly as invariant values inside the output pyramid during multi-resolution synthesis.
This is not theoretically optimal, as hard constraints should be on the highest level only.
But it seems to work well in practice.
For better theoretical approach, we will need to pursue a joint optimization schemes such as \cite{Kwatra:2005:TOE,Han:2006:TDO} or a multi-grid solver.
